\documentclass{lpscabs} 
\usepackage{times}
\usepackage{graphicx}
\usepackage[numbers]{natbib}

% \newcommand{\bibfont}{\small}
% \setlength{\bibsep}{0pt}

% If your abstract is two pages long, type your running head in the argument
% below:

\runningtitle{CONSTRAINTS ON FINE LAYERS IN GANGES AND HEBES MENSAE: 
R. A. Beyer \& A. S. McEwen}


\begin{document}

% Type your title in the argument below, using upper- and lower-case letters:

\begin{\titlearea}{Constraints on the origin of fine layers in {Ganges Mensa} and {Hebes Mensa}, {Mars}}
Ross A. Beyer and Alfred S. McEwen, Lunar and Planetary Lab, The University of Arizona, Tucson, AZ, 85721-0092, USA (rbeyer@lpl.arizona.edu)
\end{\titlearea}

\begin{abstracttext}

% paper abstract:
% Ganges Mensa and Hebes Mensa are large mesas in the Valles Marineris
% system on Mars. Their surfaces and slopes display a variety of
% layering within the mensae. These mensae have been proposed to be
% volcanic tuya with layers that dip radially outwards at a steep
% angle.  We find no evidence that such steep outward-dipping layers
% are present, in fact most are much shallower than thirty degrees.
% Observations are consistent with fluvial deposits of horizontal
% layers, or perhaps very eroded tuya.  At a few places the dip angle
% and dip direction of these very thin layers can be measured with a
% combination of MOC and Mars Orbital Laser Altimeter (MOLA) data.
% Our measurements indicate horizontal to shallow dip angles for these
% fine layers, which place some constraints on their origins.

Ganges Mensa is located near 311$^{\circ}$~E, 7.5$^{\circ}$~S, and
has been studied ever since it was first resolved with the Mariner
9 spacecraft.  \citet{1973JGR....78.4009m} indicated that this was
a finely layered deposit, and that those layers did not match the
layering seen in the canyon slopes.  \citet{1973JGR....78.4063s}
noted that the layers appeared near-horizontal.  Hebes Mensa is
located near 283$^{\circ}$~E, 1$^{\circ}$~S, and is within the
entirely closed Hebes Chasma.  Observations by the Viking spacecraft
indicated that there were light and dark layers of uniform thickness
within these mensae, and the similarity
to other layered mesas and deposits in the Valles Marineris was
noted \citep{1993JGR....9811105K}.  However, \citet{2001JGR...10623429M}
showed that some of the albedo patterns previously attributed to
layering on Ganges Mensa were not bedrock layers, but topographic
benches where dark toned aeolian material accumulated (their Fig.~26).
\citet{2004PSS...52..167K} show a similar example of dark toned
material collecting on a topographic bench on Hebes Mensa (their
Fig.~11).  Ganges and Hebes Mensae are indeed layered, but the
layers are too thin to have been resolved by cameras before the Mars
Orbital Camera (MOC) \citep{2001JGR...10623429M}.
\begin{figure}[!h]
\noindent\includegraphics[width=3in]{dipping_layers.eps}
\caption{
        \label{dipping_layers}
		These cartoons show how layers with dips steeper (\emph{a.})
		and shallower (\emph{b.}) than the overall slope are expressed 
		when the surface has a fluted morphology.
        }
\end{figure}

Many hypotheses for the origin of structures like Ganges and Hebes
Mensae have been discussed \citep[see][for a review]{1992mars.book..453L}.
One of them, deposition in a low-energy lacustrine environment
\citep{1987Icar...70..409N,2000Sci...290.1927M}, accounts for the
thin, areally extensive, bedded layers that are observed.  Layers
deposited in this fashion should be horizontal unless they were
tilted by tectonic activity.

More recently, \citet{1994JGR....99.3783L} noted that these mensae
appeared to have steep sides, and suggested similarities to Icelandic
tuya.  \citet{2001JGR...10610087C} and \citet{2004PSS...52..167K}
further explored the tuya analog for these mensae and other similar
structures in the Valles Marineris.  They found similarities in
that the easily eroded flanks of tuya may be similar to the fluted
flanks of the mensae.  \citet{2001JGR...10610087C} indicated that
Ganges and Hebes Mensae had an ideal tuya form and that the fine
layers were dipping in a down-slope direction with horizontal to
steep dips.

These studies were the first to raise the issue of these fine layers
being non-horizontal.  Many qualitative descriptions of the
orientations of these fine layers have been made, but none have
made quantitative measurements of the dip angle or dip direction.
Unfortunately, measuring such characteristics
on the flanks of the mensae is difficult.  It can be accomplished
in a few locations using data from the MOC
and the Mars Orbital Laser Altimeter (MOLA).  These measurements
are difficult because the fine layers are not very distinct.  If
MOC images are separated even by just a few hundred meters, it is
not clear which layers match between the two images.  Morphological
comparisons, shadow measurements, and individual MOLA tracks aligned
with MOC images can help to narrow parameter space and constrain
dip angle and direction.

% We were only able to make one certain dip measurement, but we were
% able to constrain the dip angles of beds in the mensae.  These
% measurements constrain the dip angle of the finest beds observed
% in the mensae which has implications for the various origin hypotheses.

~\\
\textbf{Layer Dip Angles and Surface Slope Angles}\\

A first order constraint on the dip angle of bedded units can be
made from the overall angle of the slope on which they are observed
and the surface features of that slope.
% \citet{1981PGG.419} first
% indicated that the fluting on the flanks of these mensae was shaped
% like yardangs, a landform generally carved from indurated but friable
% materials
% \citep[e.g.][]{1979JGR....84.8147W}.
% Yardangs have the shape of inverted boat-hulls, and so a simple
% model of a layered volume cut by a sloping surface with yardangs
% on it might look like Fig.~\ref{dipping_layers}.  
A layered volume cut by a sloping, fluted surface might look like
Fig.~\ref{dipping_layers}.  This figure
shows that beds which have a dip direction parallel to the dip
direction of the overall surface slope have a different morphology
when viewed from above if the beds are dipping more or less steeply
than the overall slope.  

Figure~\ref{M09/03505} shows thin layers exposed on the south face
of Ganges Mensa that are exposed in this surface.  The surface
expression of the layers resembles Fig.~\ref{dipping_layers}b , not
Fig.~\ref{dipping_layers}a, indicating that these layers are dipping
less steeply than the overall surface slope.  The slope in this
area is about $25^{\circ}$, so these layers must be dipping at an
angle less than that.
\begin{figure}[!h]
\noindent\includegraphics[width=3in]{m0903505.lev2.eps}
\caption{
        \label{M09/03505}
		This is a portion of the MOC image M09/03505, it shows thin layers 
		on the south face of Ganges Mensa that are exposed in the fluting
		that covers this surface.  North is to the top of the image,
		and the downward slope is southwards.
        }
\end{figure}

Similarly, Fig.~\ref{M03/00648} shows thin
layers exposed on the north face of Hebes Mensa.  Again, the
morphology that we observe is similar to Fig.~\ref{dipping_layers}a, 
indicating that these layers are also dipping at an angle
less than the overall surface slope.  That slope is 10 to $15^{\circ}$
for the portion of M03/00648 in Fig.~\ref{M03/00648} . 
\begin{figure}[!h]
\noindent\includegraphics[width=3in]{m0300648.lev2.eps}
\caption{
        \label{M03/00648}
		This is a portion of the MOC image M03/00648 on the north
		face of Hebes Mensa, it shows thin layers exposed in the fluting
		that covers this 10 to $15^{\circ}$ slope, showing
		that those layers must be dipping at a shallow angle.  North
		is to the top of the image and the downward slope is
		northwards.
        }
\end{figure}

~\\
\textbf{Discussion}\\

\citet{1979JGR....84.8048A} measured the overall slopes of five
Icelandic tuya and found that the mean of the steepest slopes was
$34^{\circ}\pm4^{\circ}$.  

This morphology is very common in MOC observations of the flanks
of Ganges and Hebes Mensae.  Since the slopes are generally less
than $30^{\circ}$,
we can make the generalization that most of the fine layers observed
on the flanks of these mensae are dipping at angles of less than
$30^{\circ}$.  It is important to note that this observation would
also be consistent with completely horizontal layers.  Similarly,
it would be consistent with layers that dip shallowly in a direction
parallel to the overall slope (radially outwards from the mensae),
or in a dip direction into the slope (radially inwards to the
mensae).

Since we do not find widespread evidence of layering which dips
steeply in a direction radially outwards from these mensae, then the
layers can not be analogous to the steeply dipping foreset beds
often observed in terrestrial tuya.  Furthermore, the mensae have
relatively shallow slopes overall which is inconsistent with the ideal
tuya form that has been proposed for these mensae in the past.  Even
if steeply dipping foreset beds had been eroded into the slopes
that we observe on these mensae then morphologies like that in
Fig.~\ref{dipping_layers}a should be common on the shallow flank
slopes that we observe.

These observations do not necessarily preclude the mensae from
having formed in a manner broadly similar to terrestrial tuya.  as
these characteristics could be consistent with a more severely
eroded tuya.
The steeply dipping foreset beds may have been eroded away or be
preserved in unimaged or buried locations, and the fine layering
that is observed may be from the inner core of the tuya.  Alternately,
the mensae are many times the size of terrestrial tuya, and it is
possible that subglacial edifices of this size do not have the
steeply dipping foreset beds observed on terrestrial tuya.  In this
manner, the physical processes of volcanic construction of such a
large edifice beneath and through many kilometers of ice on Mars
may have fundamental differences from those of terrestrial tuya
that we do not yet appreciate.

The finest layers in Ganges and Hebes Mensae are horizontal to
shallowly dipping.  Unfortunately, this does not particularly exclude
any of the various hypotheses for the origins of these mensae.
However, any formation hypothesis should take these measurements into
account.

\thispagestyle{empty}
\bibliography{/usr/home/rbeyer/docs/references/dissertation.bib,/usr/home/rbeyer/docs/references/layers.bib,/usr/home/rbeyer/docs/references/mgs.bib,/usr/home/rbeyer/docs/references/mo.bib,/usr/home/rbeyer/docs/references/my_abstracts.bib,/usr/home/rbeyer/docs/references/my_papers.bib,/usr/home/rbeyer/docs/references/photoclinometry.bib,/usr/home/rbeyer/docs/references/software.bib,/usr/home/rbeyer/docs/references/volcanology.bib}
\bibliographystyle{unsrtnat}

\end{abstracttext}

\end{document}

